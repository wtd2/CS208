% !TEX TS-program = xelatex
% !TEX encoding = UTF-8
% !Mode:: "TeX:UTF-8"

\documentclass[onecolumn,oneside]{SUSTechHomework}
% \usepackage{multirow,multicol,amsmath,setspace,geometry,amssymb,parskip,nccmath,proof,amsthm,graphicx}
% \usepackage[ruled,vlined]{algorithm2e}
\usepackage{amsmath,float,longtable}
% \geometry{left=2cm,right=2cm,top=2cm,bottom=2.5cm}
% \setcounter{secnumdepth}{-\maxdimen}
% \setlength{\parindent}{1em}

\author{张文灏}
\sid{11812103}
\title{Assignment 6}
\coursecode{CS208}
\coursename{Algorithm Design and Analysis}

\begin{document}
  \maketitle
  \textit{Exercise 6.2}
  \section{Exercise 6.2}
  \paragraph{(a)}
  \par Consider the following situation:
  \begin{longtable}[]{@{}cccc@{}}
  \toprule
    & Week 1 & Week 2 & Week 3\tabularnewline
    \midrule
    \endhead
    $\mathit{l}$ & 1 & 1 & 1 \tabularnewline
    $\mathit{h}$ & 1 & 3 & 100 \tabularnewline
    \bottomrule
  \end{longtable}
  \par Using the given algorithm, we will choose (no, high, low) in these $3$ weeks with the revenue $4$. However, when we choose (low, no, high) we will get the revenue of $101$, which is optimized.
  \paragraph{(b)}
  \par We can solve it by dynamic programming easily. Let $f[i][j]$ means the maximum revenue at week $i$, while $j=0$ means choosing no work at week $i$ and $j=1$ means choosing low-stress or high-stress work.
  \par Then we have the following equations
  \begin{equation*}
  \begin{aligned}
    \left\{\begin{array}{ll}f[i][0]=\max\{f[i - 1][0], f[i - 1][1]\} \\f[i][1]=\max\{f[i - 1][0] + h[i], f[i - 1][0] + l[i], f[i - 1][1] + l[i]\}\end{array}\right.
  \end{aligned}
  \end{equation*}
  with the initial value
  \begin{equation*}
  \begin{aligned}
    \left\{\begin{array}{ll}f[0][0]=-\infty \\f[0][1]=0\end{array}\right.
  \end{aligned}
  \end{equation*}
  \par Then the answer is $\max\{f[n][0], f[n][1]\}$.
\end{document}
